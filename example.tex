% Read the PDF documentation for all parameters
\documentclass[a4paper, 11pt, colorlinks]{academiccv}

\sectionSetup {
  before/color = green,
  rectangle/color = blue,
  after/color = purple,
}

\subsectionSetup {
  show-number = true,
  rectangle/width = 0,
  after/color = red,
}

\begin{document}

  {\bfseries
  This example is purposefully visually horrible in order to illustrate how to
  use the commands and their impact.
  Please read \verb!example.tex!.
  }

  \makecvtitle{
    author = John Who,
    position = Doctor,
    photo = photo,
    organization = {Secret Inc., Can't Tell},
    email = john.who@secret.com,
    email = john@canttell.anyone,
    website = john.who.me,
    github = JohnWho,
    orcid = 0000-0000-0000-0000,
    linkedIn = JohnWho,
    street = {Unknown Street, 1},
    zipcode = 987654,
    city = Middle,
    country = Nowhere,
  }

  \titleSetup{
    author = {\itshape\small},
    position = {\bfseries\Large},
    organization = {\scshape\color{blue}},
    address = {\ttfamily},
    city = {\scshape\Large},
    country = {\normalfont of\ }, % Can be used to add some visible tokens, too!
    linkedIn = {\color{blue}}, % Order is important! The next line overwrites this one
    links = {\color{green}}, % To change link colors, use hyperref commands
    github = {\color{black}},
    date = {}, % Disable showing "last update" message
    vertical-space = {2em},
    portion-photo = {.1},
  }

  \makecvtitle{
    author = John Who,
    position = Doctor,
    photo = photo,
    organization = {Secret Inc., Can't Tell},
    email = john.who@secret.com,
    email = john@canttell.anyone,
    website = john.who.me,
    github = JohnWho,
    orcid = 0000-0000-0000-0000,
    linkedIn = JohnWho,
    street = {Unknown Street, 1},
    zipcode = 987654,
    city = Middle,
    country = Nowhere,
  }

  \makecvtitle{
    author = John Who,
    position = Doctor,
    organization = {Secret Inc., Can't Tell},
    email = john.who@secret.com,
    zipcode = 987654,
    city = Middle,
    country = Nowhere,
    style = { % It is overwrite some styles in the same command
      author = {\color{green}}
    }
  }

  \section{Work experience}
  Please fill me!

  \section{Publications}

  \subsection{Published}

  \publication{
    title = {Am I a Good Man?},
    authors = {John Who, P.C.},
    reference = {WC23},
    where = {Dreaming Inside, Youth edition},
    shortWhere = {DIY},
    year = {2019},
    doi = {00.0000/000-0-000-00000-00},
    arxiv = {00.00000/arXiv.0000.00000},
    note = {Won Best Paper Award}
  }

  % It is possible to change the setup in the middle of the document.
  % The changes will only impact *all* the publications that come after
  \publicationSetup{
    title = {\itshape\large},
    authors = {\scshape},
    year = {\bfseries},
    doi = {\ttfamily},
    doi-prefix = {Link:\ },
  }

  \publication{
    title = {Oh, brilliant!},
    authors = {John Who},
    year = {2954},
    doi = {00.0000/000-0-000-00000-00},
  }

  \subsection{Preprints}
  \publication{
    title = {Exterminate!},
    authors = {Supreme Davros},
    year = {8743},
    arxiv = {0000},
    style = {  % It is possible to override the style for one specific publication
      title = {\scshape},
      arxiv-prefix = {ARXIV:\ },
    },
  }

  % If a publication overwrites some style variables, they are restored to their
  % previous value for the next publications (i.e., the overwrites are local)
  \publication{
    title = {Are you my mommy?},
    authors = {Some Kids},
    year = {1942},
  }

  \subsubsection{Subsubsection}

  \section{Work Experience}

  \job{
    start = 1963,
    end = 1966,
    title = The granddad,
  }
  \job{
    start = 1966,
    end = 1969,
    title = The clown,
    description = {Stop, you're making me giddy!},
    organization = Second body,
  }
  \job{
    start = 1970,
    end = 1974,
    title = The exiled,
    description = {Ungh\dots Shoes, must find my shoes.},
    organization = Third body,
    style = { % use \jobSetup to setup style globally
      start = {\itshape},
      end = {\itshape},
      title = {\scshape},
      description = {\bfseries},
      organization = {\Large},
    },
  }
  \job {
    start = 1974,
    end = 1981,
    title = The scarf,
    description = Would you like a jelly baby?,
    organization = Fourth body,
    style = {
      swap,
    },
  }

  \section{Talks}

  \talk{
    title = Let's discuss the future,
    date = 20 January 2765,
    conference = FUTURE,
    where = Earth
  }

  \talk{
    title = Let's discuss the future,
    date = 20 January 2765,
    conference = FUTURE,
    where = Earth,
    style = { % use \talkSetup for global settings
      swap,
      title = {\scshape},
      date = {\itshape},
      conference = {\ttfamily},
      where = {\color{blue}},
    },
  }

  \section{Teaching}

  \teach{
    year = {Academic years 2021 to 2024},
    course = {General Physics},
    role = Associate Professor,
    level = First year,
    organization = Somewhere,
    description = Main teacher.,
  }

  \teach{
    year = {Academic years 2021 to 2024},
    course = {Time and Stuff},
    role = Teaching assistant,
    level = {Twelfth year},
    organization = Somewhere else,
    description = Teaching assistant supervising the practical sessions.,
    style = { % use \teachSetup for global settings
      swap,
      year = {\itshape},
      course = {\scshape},
      role = {\bfseries},
      organization = {\slshape},
      description = {\ttfamily}
    }
  }

  \section{Supervision}

  \supervision{
    year = Summer of 1970,
    name = Liz Shaw,
    role = Supervisor,
    organization = Top Secret,
    description = First assistant of the third body.
  }

  \supervision{
    year = 1971-1972,
    name = Jo Grant,
    role = Supervisor,
    organization = Top Secret,
    description = Second assistant of the third body.,
    style = { % use \supervisionSetup for global settings
      swap,
      year = {\itshape},
      name = {\scshape},
      role = {\ttfamily},
      organization = {\tiny},
      description = {\bfseries},
    },
  }

  \section{Project}

  \project{
    shortName = Sonic,
    name = Sonic Screwdriver,
    role = Conceptor,
    description = It makes noise!,
    homepage = {https://github.com/DocSkellington/academic-cv},
    artifact = {https://zenodo.org/},
  }

  \project{
    shortName = SonicSun,
    name = Sonic Sunglasses,
    role = Conceptor,
    description = It makes noise \emph{and I look cool}!,
    style = { % use \projectSetup for global settings
      swap,
      shortName = {\scshape},
      name = {\itshape},
      role = {\color{blue}},
      description = {\bfseries}
    }
  }
  
  \project{
    name = Long description test,
    description = {Here is an itemize description of this project:
      \begin{itemize}
        \item Item number 1
        \item Item number 2
      \end{itemize}
    }
  }

  \section{Awards}

  \award{
    year = 2019,
    description = Best paper award for \emph{Am I a Good Man?},
  }

  \award{
    year = 2015,
    description = {Some random award. Honestly, I forgot.},
    style = {
      swap,
      year = {\itshape},
      description = {\slshape},
    }
  }

  \section{CV Items}
  
  In case the modules do not cover one of your needs, you can use generic CV items.

  \cvitem{Margin}{Simple content}

  \cvitem[
    margin={\bfseries},
    main={\scshape},
  ]{
    Margin\\
    {\small Smaller margin}
  }{
    Content\\
    {\small Some smaller text}
  }

  \cvitem[swap]{Margin}{Simple content}

  \cvitem[swap]{
    Margin\\
    {\small Smaller margin}
  }{
    Content\\
    {\small Some smaller text}
  }
\end{document}